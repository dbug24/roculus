\documentclass[a4paper, 12pt]{article}
\usepackage[utf8]{inputenc}
\usepackage[T1]{fontenc}
\usepackage[margin=1.0in]{geometry}

\title{How to use the Roculus visualization for ROSIE.}
\date{2014-08-21}
\author{Daniel Bug}

\begin{document}
\maketitle

\section{Setting up ROSIE}
Here are the different steps to be done on the robot before the application start:
\begin{itemize}
\item Turn on the key and start-up robot base and computer.
\item When the login screen shows up, go into the EBC switch menu on the robot base and turn the last two 24.0V outputs ON. (7-0 and 7-1, 24V).
\item Switch on the sidekick computer by pressing the small red button at the side of the terminal.
\item Open a console and begin to start up the robot operating system. There is a startup script for a \emph{tmux} session that will have a set of all interesting terminals for you: \texttt{Ctrl+R 'start.sh' -> }. It takes a while to start teh first time. The first terminal is a display of the running processes (htop) and you can browse the terminals by using \texttt{Ctrl+b, n} or \texttt{Ctrl+b, p} to get to the next/previous terminal.\\ The following things need to run:
  \begin{itemize}
  \item \texttt{rosie\_core}
  \item \texttt{rosie\_robot}
  \item \texttt{rosie\_navigation} (You will probably have to init the navigation in rviz manually to have to robot localized correctly.)
  \item \texttt{rosie\_head\_camera}, which needs to be run on the sidekick. (Check that the terminal is an ssh session on the strands-sidekick, before you start the camera launch file).
  \end{itemize}
\item The application does not listen to the pure camera topics, but - in order to save bandwidth - relies on topics that are republished from a \emph{drop} node (\texttt{Ctrl+R 'drop'}):
  \begin{itemize}
  \item Open a new (tmux) terminal by typing \texttt{Ctrl+b, c}
  \item drop-node for the rgb images: \texttt{rosrun topic\_tools drop \\/head\_xtion/rgb/image\_color/compressed 9 10 \\/head\_xtion/rgb/image\_color/reducedBW/compressed \&}
  \item drop-node for the depth images: \texttt{rosrun topic\_tools drop\\ /head\_xtion/depth/hw\_registered/image\_rect/compressedDepth 9 10 \\/head\_xtion/depth\_registered/image\_rect/reducedBW/compressedDepth \&}
  \end{itemize}
The \texttt{9 10} parameters indicate that we will drop 9 out of 10 images, which will give around 3 fps for the video stream. You can work with higer values, but there will be a limit at which you start trading off a better frame rate against the position and orientation updates. For dropping 2/3 e.g. the images might take to much bandwidth for the transforms to be updated regularly and the stream will look nice, but appear at the wrong place.\\
The topics should print a notification on the terminal saying that they are advertising the new topic. If you don't see the message the head camera was probably not started correctly. There might be a short-cut, but I used to kill all terminals and launch everything from scratch again.
\end{itemize}

\section{Starting the Application}
Roculus relies on the folder structure. During start-up the roculus directory will be searched for multiple config files and map-components. They are explained in the next section.
\begin{itemize}
\item To start the application:
  \begin{itemize}
  \item On the visualization PC, make sure you are connected to rosienet and that your \texttt{/etc/hosts} file lists the correct IP adresses.
  \item Check with \texttt{echo \$ROS\_MASTER\_URI} that this environment variable is pointing to the robot, i.e.\ \texttt{http://scitosstrands:11311}
  \item If you want to use the gamepad launch the teleoperation node in a separate terminal:\\\texttt{roslaunch scitos\_teleop teleop\_joystick.launch}
  \item Go to the roculus folder: \texttt{roscd roculus}
  \item Start roculus: \texttt{rosrun roculus roculus\_node}, the start-up time can be in minutes, if the program is loading multiple room scans into the environment. (You can check the terminal output. To get there use \texttt{Alt+Tab}).
  \end{itemize}
\end{itemize}

\section{Configs, Resources and Map Data}
As mentioned there are several configs and resources used by the application:
\begin{itemize}
\item ogre.cfg: contains the screen and render settings for ogre, if this file can not be read, ogre will show a config dialog to recreate such a file.
\item plugins.cfg: specifies the plugins that will be loaded on start-up. Probably just the cg and particle library is needed, the rest will be commented out.
\item resources.cfg: points to the materials/textures/shaders for the game:
  \begin{itemize}
  \item media/sibenik.zip: one image from this archive is used as default initialization for textures
  \item media/rosie.zip: everything for the robot avatar
  \item media/game.zip: everything for the game... meshes for the keys/treasure/locks, their materials, etc.
  \item media/vertexColor.material: contains the materials for the thesis application, (blank material, video stream texture, snapshot texture)
  \item media/projection3D.cg: the cg shader programs that actually perform the reprojection from camera geometry to 3D snapshot. The color transformation for the sepia look if defined here as well.
  \end{itemize}
\item map directory: using the simpleXMLparser from Rares and the multiroom parser, all patrol-recordings in this folder are loaded into the environment during start-up
\item game.cfg: the configuration of your game environment. How many waypoints, how many keys, what are the rooms/corridors, which waypoints can be used for keys/locks/etc. Take a look at the comments inside the file.
\end{itemize}

\section{Functions on the Keyboard}
The keyboard will be used as a supervisor input. It can:
\begin{itemize}
\item ESC (3 times): End the application
\item i: reinitialize the game
\item p: toggle first person mode
\item a,d,s,w,(arrows), PgUp, PgDn: Move around in the world in free view
\item SPC: select a navigation target
\item m: toggle visibility of the map
\item v: toggle the visibility of the preloaded environment
\item F3: toggle visibility of the frame rate display
\item F4: toggle visibility of the application info display
\end{itemize}

\section{Functions on the Gamepad and Mouse\\(Player Input)}
The mouse can only select navigation targets by clicking (the selection still involves looking at the waypoint).\\
The gamepad has 4 possible functions:
\begin{itemize}
\item buttons 1-4: select a navigation target
\item button 8: reset the oculus orientation to look in the direction of the robot
\item cross-joystick (x-direction): look 120deg behind you (to the left/right)
\item only if the supervisor unmouted the player from first-person: you can fly around with the joysticks (left: forward+backward/turn, right: up+down and step left/right)
\end{itemize}

\end{document}
